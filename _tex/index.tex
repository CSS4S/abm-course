% Options for packages loaded elsewhere
\PassOptionsToPackage{unicode}{hyperref}
\PassOptionsToPackage{hyphens}{url}
\PassOptionsToPackage{dvipsnames,svgnames,x11names}{xcolor}
%
\documentclass[
  letterpaper,
  DIV=11,
  numbers=noendperiod]{scrartcl}

\usepackage{amsmath,amssymb}
\usepackage{iftex}
\ifPDFTeX
  \usepackage[T1]{fontenc}
  \usepackage[utf8]{inputenc}
  \usepackage{textcomp} % provide euro and other symbols
\else % if luatex or xetex
  \usepackage{unicode-math}
  \defaultfontfeatures{Scale=MatchLowercase}
  \defaultfontfeatures[\rmfamily]{Ligatures=TeX,Scale=1}
\fi
\usepackage{lmodern}
\ifPDFTeX\else  
    % xetex/luatex font selection
\fi
% Use upquote if available, for straight quotes in verbatim environments
\IfFileExists{upquote.sty}{\usepackage{upquote}}{}
\IfFileExists{microtype.sty}{% use microtype if available
  \usepackage[]{microtype}
  \UseMicrotypeSet[protrusion]{basicmath} % disable protrusion for tt fonts
}{}
\makeatletter
\@ifundefined{KOMAClassName}{% if non-KOMA class
  \IfFileExists{parskip.sty}{%
    \usepackage{parskip}
  }{% else
    \setlength{\parindent}{0pt}
    \setlength{\parskip}{6pt plus 2pt minus 1pt}}
}{% if KOMA class
  \KOMAoptions{parskip=half}}
\makeatother
\usepackage{xcolor}
\setlength{\emergencystretch}{3em} % prevent overfull lines
\setcounter{secnumdepth}{-\maxdimen} % remove section numbering
% Make \paragraph and \subparagraph free-standing
\ifx\paragraph\undefined\else
  \let\oldparagraph\paragraph
  \renewcommand{\paragraph}[1]{\oldparagraph{#1}\mbox{}}
\fi
\ifx\subparagraph\undefined\else
  \let\oldsubparagraph\subparagraph
  \renewcommand{\subparagraph}[1]{\oldsubparagraph{#1}\mbox{}}
\fi


\providecommand{\tightlist}{%
  \setlength{\itemsep}{0pt}\setlength{\parskip}{0pt}}\usepackage{longtable,booktabs,array}
\usepackage{calc} % for calculating minipage widths
% Correct order of tables after \paragraph or \subparagraph
\usepackage{etoolbox}
\makeatletter
\patchcmd\longtable{\par}{\if@noskipsec\mbox{}\fi\par}{}{}
\makeatother
% Allow footnotes in longtable head/foot
\IfFileExists{footnotehyper.sty}{\usepackage{footnotehyper}}{\usepackage{footnote}}
\makesavenoteenv{longtable}
\usepackage{graphicx}
\makeatletter
\def\maxwidth{\ifdim\Gin@nat@width>\linewidth\linewidth\else\Gin@nat@width\fi}
\def\maxheight{\ifdim\Gin@nat@height>\textheight\textheight\else\Gin@nat@height\fi}
\makeatother
% Scale images if necessary, so that they will not overflow the page
% margins by default, and it is still possible to overwrite the defaults
% using explicit options in \includegraphics[width, height, ...]{}
\setkeys{Gin}{width=\maxwidth,height=\maxheight,keepaspectratio}
% Set default figure placement to htbp
\makeatletter
\def\fps@figure{htbp}
\makeatother
% definitions for citeproc citations
\NewDocumentCommand\citeproctext{}{}
\NewDocumentCommand\citeproc{mm}{%
  \begingroup\def\citeproctext{#2}\cite{#1}\endgroup}
\makeatletter
 % allow citations to break across lines
 \let\@cite@ofmt\@firstofone
 % avoid brackets around text for \cite:
 \def\@biblabel#1{}
 \def\@cite#1#2{{#1\if@tempswa , #2\fi}}
\makeatother
\newlength{\cslhangindent}
\setlength{\cslhangindent}{1.5em}
\newlength{\csllabelwidth}
\setlength{\csllabelwidth}{3em}
\newenvironment{CSLReferences}[2] % #1 hanging-indent, #2 entry-spacing
 {\begin{list}{}{%
  \setlength{\itemindent}{0pt}
  \setlength{\leftmargin}{0pt}
  \setlength{\parsep}{0pt}
  % turn on hanging indent if param 1 is 1
  \ifodd #1
   \setlength{\leftmargin}{\cslhangindent}
   \setlength{\itemindent}{-1\cslhangindent}
  \fi
  % set entry spacing
  \setlength{\itemsep}{#2\baselineskip}}}
 {\end{list}}
\usepackage{calc}
\newcommand{\CSLBlock}[1]{\hfill\break\parbox[t]{\linewidth}{\strut\ignorespaces#1\strut}}
\newcommand{\CSLLeftMargin}[1]{\parbox[t]{\csllabelwidth}{\strut#1\strut}}
\newcommand{\CSLRightInline}[1]{\parbox[t]{\linewidth - \csllabelwidth}{\strut#1\strut}}
\newcommand{\CSLIndent}[1]{\hspace{\cslhangindent}#1}

\KOMAoption{captions}{tableheading}
\makeatletter
\@ifpackageloaded{caption}{}{\usepackage{caption}}
\AtBeginDocument{%
\ifdefined\contentsname
  \renewcommand*\contentsname{Table of contents}
\else
  \newcommand\contentsname{Table of contents}
\fi
\ifdefined\listfigurename
  \renewcommand*\listfigurename{List of Figures}
\else
  \newcommand\listfigurename{List of Figures}
\fi
\ifdefined\listtablename
  \renewcommand*\listtablename{List of Tables}
\else
  \newcommand\listtablename{List of Tables}
\fi
\ifdefined\figurename
  \renewcommand*\figurename{Figure}
\else
  \newcommand\figurename{Figure}
\fi
\ifdefined\tablename
  \renewcommand*\tablename{Table}
\else
  \newcommand\tablename{Table}
\fi
}
\@ifpackageloaded{float}{}{\usepackage{float}}
\floatstyle{ruled}
\@ifundefined{c@chapter}{\newfloat{codelisting}{h}{lop}}{\newfloat{codelisting}{h}{lop}[chapter]}
\floatname{codelisting}{Listing}
\newcommand*\listoflistings{\listof{codelisting}{List of Listings}}
\makeatother
\makeatletter
\makeatother
\makeatletter
\@ifpackageloaded{caption}{}{\usepackage{caption}}
\@ifpackageloaded{subcaption}{}{\usepackage{subcaption}}
\makeatother
\ifLuaTeX
  \usepackage{selnolig}  % disable illegal ligatures
\fi
\usepackage{bookmark}

\IfFileExists{xurl.sty}{\usepackage{xurl}}{} % add URL line breaks if available
\urlstyle{same} % disable monospaced font for URLs
\hypersetup{
  pdftitle={EBS 182/282: Agent-based modeling for sustainability},
  colorlinks=true,
  linkcolor={blue},
  filecolor={Maroon},
  citecolor={Blue},
  urlcolor={Blue},
  pdfcreator={LaTeX via pandoc}}

\title{EBS 182/282: Agent-based modeling for sustainability}
\author{}
\date{}

\begin{document}
\maketitle

\subsubsection{Contact information}\label{contact-information}

Instructor: Matt Turner {[}\href{https://mt.digital}{web}{]}

Email: \url{maturner@stanford.edu}

Office hours: 1-3 PM Monday and Tuesday in Y2E2 357.

Also available by appointment for in-person or Zoom meetings.

\subsubsection{Logistics}\label{logistics}

Class times: MW 11:30 - 12:50, Spring 2025

Course website: \url{https://css4s.github.io/abm-course}

\subsubsection{Motivation}\label{motivation}

Sustainable practices are useless if no one practices them. The
development and diffusion of sustainable practices requires a mix of
individual experimentation, social learning and influence, and
cooperation and coordination, just to name a few social behaviors. It is
essential, then, to develop rigorous, flexible, and practical methods
for the simulation of social systems. Students in this course will learn
how to use agent-based model simulations of social diffusion of
sustainable practices, just like an engineer might first develop
computer models to design new products, which are then made into
prototypes, and eventually built and delivered to customers.
\emph{Sustainable practices} here could be taken as any behavior that
advances progress towards achieving \href{https://sdgs.un.org/goals}{UN
Sustainable Development Goals}.

\subsubsection{Learning goals}\label{learning-goals}

Students who successfully complete the course will learn:

\begin{itemize}
\tightlist
\item
  To understand the theoretical origins of agent-based models of social
  behavior and how to put them to use to support decision making to
  promote sustainability.
\item
  To write computational social science research papers, including
  specific tips for describing agent-based models.
\item
  Software development skills, such as the ability to write and organize
  R functions and scripts for more sustainable \emph{software}
  practices. Software development skills like these make research more
  reproducible, efficient, and fun.
\end{itemize}

\subsubsection{Prerequisites}\label{prerequisites}

The course is designed so that students with minimal programming
experience can succeed. Students with no programming experience may feel
lost unless extra time is invested. In any case, I am happy to help
students with any level of programming experience to elevate their
skills, but it is the student's responsibility to seek out help. We will
be programming in R using RStudio on student laptops and also running
our models on the \href{https://srcc.stanford.edu/farmshare}{Farmshare
computing cluster} provided for educational purposes at Stanford.

Some experience with mathematics, especially probability and statistics,
will also help, but is not necessary as long as a student in need of
help seeks it out. I am happy to help students succeed with any level of
mathematical experience.

\subsubsection{Coursework and grading}\label{coursework-and-grading}

\begin{itemize}
\tightlist
\item
  Four problem sets (10\% each, 40\% total)
\item
  Midterm project (20\%)
\item
  Final project (40\%)
\end{itemize}

Attendance is essential for success in this course. This course is a
hybrid lecture-lab course, where about half of class time will be
devoted to lecture and half to lab work where I will be available for
individual help and during which we will sometimes be assigned to groups
to discuss course material or work together on problem sets. Whether a
given course meeting is a lecture or lab will depend on the rate of our
progress through the different subjects and student interests and needs.

The four problem sets will tentatively follow this sequence:

\begin{enumerate}
\def\labelenumi{\arabic{enumi}.}
\tightlist
\item
  Introduction to agent-based modeling and computational social science
  for sustainability
\item
  Computational experiment infrastructure: cluster computing and data
  management for agent-based modeling
\item
  Reinforcement learning agents, uncertain environments, and selection
  for social learning
\item
  Cooperation, coordination, social influence, and dynamic social
  networks
\end{enumerate}

The midterm project is just a first draft of the final project, so in
that sense they are one big project. In these projects, students will
write their own research papers by following a detailed outline to
provide all necessary components. I will work with students during
in-class project workshops and in office hours or other out of class
time to identify a project of interest and develop the modeling approach
to make a real contribution to understanding their chosen problem.

\paragraph{Accessibility}\label{accessibility}

Students who need an academic accommodation based on the impact of a
disability must initiate the request with the Office of Accessible
Education (OAE). Professional staff will evaluate the request with
required documentation, recommend reasonable accommodations, and prepare
an Accommodation Letter for faculty. Unless the student has a temporary
disability, Accommodation letters are issued for the entire academic
year. Students should contact the OAE as soon as possible since timely
notice is needed to coordinate accommodations. The OAE is located at 563
Salvatierra Walk (phone: 650-723-1066).

\paragraph{Late work policy}\label{late-work-policy}

Unless we discuss otherwise ahead of time, the late work policy is as
follows.

\begin{itemize}
\tightlist
\item
  Problem sets: 50\% credit up to one week late.
\item
  Midterm: 50\% credit up to one week late.
\item
  Final: No credit.
\end{itemize}

\paragraph{Course technology and
materials}\label{course-technology-and-materials}

\begin{itemize}
\tightlist
\item
  Students will use the R programming language, including the
  \href{https://css4s.github.io/socmod/}{\texttt{socmod}} library I am
  developing for this and a suite of courses in Computational Social
  Science for Sustainability (CSS4S) and for research use.
\item
  Readings will come from journal articles and course notes that will be
  periodically released.
\item
  Students will receive detailed in-person guidance on how to develop,
  execute, and communicate their own social science for sustainability
  projects using agent-based modeling.
\end{itemize}

\subsubsection{Expectations and
policies}\label{expectations-and-policies}

You can expect me to be there for you to help you develop programming
skills to understand and complete the problem sets. You can expect me to
respond promptly and attentively to your needs in the course. This
course is programming-intensive: you'll be expected to use and write
functions in R and use some more advanced programming concepts.

I expect students in the course to seek help early if they encounter
difficulties. I am happy to make extra time outside the listed office
hours to bring students up to speed, in person or on Zoom.

If life events make finishing the problem sets difficult, I am happy to
discuss accommodations. However, these needs must be communicated before
the deadline. Please don't hesitate to ask for help in any way before or
after a deadline. I know life happens and I will try to find a way to
facilitate your success in the course.

\subsubsection{Course outline}\label{course-outline}

The agent-based approach allows social scientists to synthesize social
network and cognitive models to represent social behavior structured by
our dynamic social relationships. Simulations of \emph{interventions} to
promote sustainable practices can be used to select from candidate sets
of interventions in different real-world social networks. When reality
inevitably does not match predictions, this provides motivation and
guidance to empirical studies that can help infer potential mechanisms
inhibiting the adoption of sustainable practices. These mechanisms can
be included in future model predictions, which again motivates further
empirical refinement.

This course will teach students how to use agent-based models to
simulate the process of social diffusion in diverse contexts with
different behaviors including exploration, social learning and
influence, and cooperation and coordination. Developing this expertise
requires developing and combining computing skills and a foundational
knowledge of relevant social and behavioral science.

\begin{enumerate}
\def\labelenumi{\arabic{enumi}.}
\tightlist
\item
  Introduction to computational social science for sustainability and
  agent based modeling.
\item
  Computational infrastructure for agent-based modeling. Agent-based
  modeling requires some specific computational techniques: it is
  necessary to write one's own functions and use higher-order functions;
  wrangle data; and plot simulation outputs to do agent-based modeling.
  For publication-quality research it is often also necessary to use
  hundreds or more computer processors at a time, enabled by using
  shared supercomputer clusters, like
  \href{https://srcc.stanford.edu/farmshare}{Famrshare for educational
  purposes} or \href{https://www.sherlock.stanford.edu/}{Sherlock for
  research} at Stanford. Students will learn these techniques and good
  programming practices including documentation and ``literate
  programming'' that uses informative variable names.
\item
  Social learning, whose utility is sensitive to ecological uncertainty,
  since variability or unpredictability can reduce the value of social
  information (Turner, Moya, et al. 2023). The adaptive mixing of social
  and asocial information enables humans to optimize behavioral choices,
  as observed in experiments complemented by agent-based models with
  reinforcement learning agents (Witt et al. 2024). Social reinforcement
  learning agents are actively being developed as a strategy to harness
  advantages of social learning for artificial intelligence systems
  (Jaques et al. 2019; Ndousse et al. 2021).
\item
  Group structure in human relationships and its emergence via
  \emph{homophily}, i.e., the phenomenon for people to more frequently
  interact within their own group versus between groups (McPherson,
  Smith-Lovin, and Cook 2001), which could be by choice or imposed by
  socio-economic or other circumstance (Kossinets and Watts 2009). This
  can create echo chambers and promote between-group animosity but can
  also provide a substrate for greater innovation and diffusion of
  adaptations compared to homogeneous populations (Centola 2010, 2011;
  Turner, Singleton, et al. 2023). Social networks are dynamic: their
  evolution may promote adaptation \emph{or} maladaptation (Smolla and
  Akçay 2019; Centola 2022).
\item
  Throughout presentation of course material, students will learn how
  computational models of social behavior could be used for
  understanding real-world phenomena, with a focus on their use in
  decision making to support sustainability.
\end{enumerate}

\subsubsection*{References}\label{references}
\addcontentsline{toc}{subsubsection}{References}

\phantomsection\label{refs}
\begin{CSLReferences}{1}{0}
\bibitem[\citeproctext]{ref-Centola2010}
Centola, Damon. 2010. {``{The Spread of Behavior in an Online Social
Network Experiment}.''} \emph{Science} 329 (5996): 1194--97.
\url{https://doi.org/10.1126/science.1185231}.

\bibitem[\citeproctext]{ref-Centola2011}
---------. 2011. {``{An experimental study of homophily in the adoption
of health behavior}.''} \emph{Science} 334 (6060): 1269--72.
\url{https://doi.org/10.1126/science.1207055}.

\bibitem[\citeproctext]{ref-Centola2022}
---------. 2022. {``{The network science of collective intelligence}.''}
\emph{Trends in Cognitive Sciences} 26 (11): 923--41.
\url{https://doi.org/10.1016/j.tics.2022.08.009}.

\bibitem[\citeproctext]{ref-Jaques2019}
Jaques, Natasha, Angeliki Lazaridou, Edward Hughes, Caglar Gulcehre,
Pedro A Ortega, DJ Strouse, Joel Z Leibo, and Nando de Freitas. 2019.
{``{Social Influence as Intrinsic Motivation for Multi-Agent Deep
Reinforcement Learning}.''} In \emph{Proceedings of the 36th
International Conference on Machine Learning}, 10.
\url{https://arxiv.org/abs/1810.08647}.

\bibitem[\citeproctext]{ref-Kossinets2009}
Kossinets, Gueorgi, and Duncan J. Watts. 2009. {``{Origins of homophily
in an evolving social network}.''} \emph{American Journal of Sociology}
115 (2): 405--50. \url{https://doi.org/10.1086/599247}.

\bibitem[\citeproctext]{ref-McPherson2001}
McPherson, Miller, Lynn Smith-Lovin, and James M Cook. 2001. {``{Birds
of a Feather: Homophily in Social Networks}.''} \emph{Annual Review of
Sociology} 27 (1): 415--44.
\url{https://doi.org/10.1146/annurev.soc.27.1.415}.

\bibitem[\citeproctext]{ref-Ndousse2021}
Ndousse, Kamal, Douglas Eck, Sergey Levine, and Natasha Jaques. 2021.
{``{Emergent Social Learning via Multi-agent Reinforcement Learning}.''}
In \emph{Proceedings of the 38th International Conference on Machine
Learning}. \url{http://arxiv.org/abs/2010.00581}.

\bibitem[\citeproctext]{ref-Smolla2019}
Smolla, Marco, and Erol Akçay. 2019. {``{Cultural selection shapes
network structure}.''} \emph{Science Advances} 5 (8).
\url{https://doi.org/10.1126/sciadv.aaw0609}.

\bibitem[\citeproctext]{ref-Turner2023a}
Turner, Matthew A., Cristina Moya, Paul E. Smaldino, and James Holland
Jones. 2023. {``{The form of uncertainty affects selection for social
learning}.''} \emph{Evolutionary Human Sciences} 5.
\url{https://doi.org/10.1017/ehs.2023.11}.

\bibitem[\citeproctext]{ref-Turner2023}
Turner, Matthew A., Alyson L. Singleton, Mallory J. Harris, Ian
Harryman, Cesar Augusto Lopez, Ronan Forde Arthur, Caroline Muraida, and
James Holland Jones. 2023. {``{Minority-group incubators and
majority-group reservoirs support the diffusion of climate change
adaptations}.''} \emph{Philosophical Transactions of the Royal Society
B: Biological Sciences} 378 (1889).
\url{https://doi.org/10.1098/rstb.2022.0401}.

\bibitem[\citeproctext]{ref-Witt2024}
Witt, Alexandra, Wataru Toyokawa, Kevin N. Lala, Wolfgang Gaissmaier,
and Charley M. Wu. 2024. {``{Humans flexibly integrate social
information despite interindividual differences in reward}.''}
\emph{Proceedings of the National Academy of Sciences} 121 (39).
\url{https://doi.org/10.1073/pnas.2404928121/-/DCSupplemental.Published}.

\end{CSLReferences}



\end{document}
